\documentclass[12pt, a4paper, oneside]{ctexbook}



%----------各种package----------%
\usepackage{amsmath, amsthm, amssymb, bm, graphicx, hyperref, mathrsfs}
%%-----设置section样式-----%%
\CTEXsetup[format={\Large\bfseries}]{section}                        %设置章标题字号为Large,居左
\CTEXsetup[number={\chinese{section}}]{section}                      %section形式改为一,二,三,..
 
\CTEXsetup[name={(,)}]{subsection}                                 
\CTEXsetup[number={\chinese{subsection}}]{subsection}                %subsection形式改为(一,二,三,...)
                            
\CTEXsetup[number=\arabic{subsubsection}]{subsubsection}    %subsubsection形式改为1,2,3,..
%%-----设置section样式-END-----%%

\usepackage{appendix} %附录
%----------各种package-END----------%




\title{{\Huge{\textbf{笔记本标题}}}\\——副标题}
\author{Znamya}
\date{\today}
\linespread{1.5}
\newtheorem{theorem}{定理}[section]
\newtheorem{definition}[theorem]{定义}
\newtheorem{lemma}[theorem]{引理}
\newtheorem{corollary}[theorem]{推论}
\newtheorem{example}[theorem]{例}
\newtheorem{proposition}[theorem]{命题}

\begin{document}

\maketitle

\pagenumbering{roman}
\setcounter{page}{1}

\begin{center}
    \Huge\textbf{前言}
\end{center}~\

这是笔记的前言部分. 
~\\
\begin{flushright}
    \begin{tabular}{c}
        Znamya\\
        \today
    \end{tabular}
\end{flushright}

\newpage
\pagenumbering{Roman}
\setcounter{page}{1}
\tableofcontents
\newpage
\setcounter{page}{1}
\pagenumbering{arabic}

\chapter{章节标题}

在这里可以输入笔记的内容. English


\section{小节标题}

这是笔记的正文部分. 

\bibliography{mylib}




\begin{appendices}
\CTEXsetup[number=\Alph{section}]{section} 
\section{xxx}
附录内容
\section{xxx}
...
\end{appendices}
\end{document}